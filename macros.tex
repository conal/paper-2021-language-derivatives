\RequirePackage{xcolor}

\newcommand\nc\newcommand
\nc\rnc\renewcommand

\nc\out[1]{}

%% \nc\noteOut[2]{\note{#1}\out{#2}}

%% To redefine for a non-draft
\nc\indraft[1]{#1}

\nc\note[1]{\indraft{\textcolor{red}{#1}}}

\nc\notefoot[1]{\note{\footnote{\note{#1}}}}

\nc\todo[1]{\note{To do: #1}}

\nc\eqnlabel[1]{\label{equation:#1}}
\nc\eqnref[1]{Equation~\ref{equation:#1}}
\nc\eqnreftwo[2]{Equations~\ref{equation:#1} and \ref{equation:#2}}

\nc\figlabel[1]{\label{fig:#1}}
\nc\figref[1]{Figure~\ref{fig:#1}}
\nc\figreftwo[2]{Figures~\ref{fig:#1} and \ref{fig:#2}}

\nc\seclabel[1]{\label{sec:#1}}
\nc\secref[1]{Section~\ref{sec:#1}}
\nc\secreftwo[2]{Sections~\ref{sec:#1} and~\ref{sec:#2}}
\nc\secrefs[2]{Sections \ref{sec:#1} through \ref{sec:#2}}

\nc\appref[1]{Appendix~\ref{sec:#1}}

%% The name \secdef is already taken
\nc\sectiondef[1]{\section{#1}\seclabel{#1}}
\nc\subsectiondef[1]{\subsection{#1}\seclabel{#1}}
\nc\subsubsectiondef[1]{\subsubsection{#1}\seclabel{#1}}

\nc\needcite{\note{[ref]}}

\nc\sectionl[1]{\section{#1}\seclabel{#1}}
\nc\subsectionl[1]{\subsection{#1}\seclabel{#1}}

\nc\workingHere{
\vspace{1ex}
\begin{center}
\setlength{\fboxsep}{3ex}
\setlength{\fboxrule}{4pt}
\huge\textcolor{red}{\framebox{Working here}}
\end{center}
\vspace{1ex}
}

%% For multiple footnotes at a point. Adapted to recognize \notefoot as well
%% as \footnote. See https://tex.stackexchange.com/a/71347,
\let\oldFootnote\footnote
\nc\nextToken\relax
\rnc\footnote[1]{%
    \oldFootnote{#1}\futurelet\nextToken\isFootnote}
\nc\footcomma[1]{\ifx#1\nextToken\textsuperscript{,}\fi}
\nc\isFootnote{%
    \footcomma\footnote
    \footcomma\notefoot
}

% Arguments: env, label, caption, body
\nc\figdefG[4]{\begin{#1}[tbp]
\begin{center}
#4
\end{center}
\caption{#3}
\figlabel{#2}
\end{#1}}

% Arguments: label, caption, body
\nc\figdef{\figdefG{figure}}
\nc\figdefwide{\figdefG{figure*}}

% Arguments: label, caption, body
\nc\figrefdef[3]{\figref{#1}\figdef{#1}{#2}{#3}}

\usepackage{hyperref}

\definecolor{mylinkcolor}{rgb}{0,0.4,0}
\hypersetup{
  linkcolor  = mylinkcolor,
  citecolor  = mylinkcolor,
  urlcolor   = mylinkcolor,
  colorlinks = false, % true
}

    % Hack to get the amsthm package working.
    % https://tex.stackexchange.com/questions/130491/xelatex-error-paragraph-ended-before-tempa-was-complete
    \let\AgdaOpenBracket\[\let\AgdaCloseBracket\]
    \RequirePackage{fontspec}
    \let\[\AgdaOpenBracket\let\]\AgdaCloseBracket

    \setmainfont
      [ Ligatures      = TeX
      , BoldItalicFont = XITS-BoldItalic.otf
      , BoldFont       = XITS-Bold.otf
      , ItalicFont     = XITS-Italic.otf
      ]
      {XITS-Regular.otf}

    %% %% \setmathfont{XITS-Math.otf}
    \setmathfont{XITSMath-Regular.otf}
    %% \setmonofont[Mapping=tex-text]{FreeMono.otf}

    % Make mathcal and mathscr appear as different.
    % https://tex.stackexchange.com/questions/120065/xetex-what-happened-to-mathcal-and-mathscr
    \setmathfont[range={\mathcal,\mathbfcal},StylisticSet=1]{XITSMath-Regular.otf}
    \setmathfont[range=\mathscr]{XITSMath-Regular.otf}


\usepackage[round]{natbib} % [square]
\bibliographystyle{plainnat}
%% \bibliographystyle{ACM-Reference-Format}

\nc\stdlibCitet[1]{\citet[\AM{#1}]{agda-stdlib}}
\nc\stdlibCitep[1]{\citep[\AM{#1}]{agda-stdlib}}
\nc\stdlibCite\stdlibCitep


\nc\agda[2]{\ExecuteMetaData[#1.tex]{#2}}
\nc\Language{\agda{Language}}
